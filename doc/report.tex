\documentclass{article}

\usepackage[utf8]{inputenc}
\usepackage[russian]{babel}
\usepackage[a4paper, margin=1in]{geometry}
\usepackage{graphicx}
\usepackage{amsmath}
\usepackage{wrapfig}
\usepackage{multirow}
\usepackage{mathtools}
\usepackage{pgfplots}
\usepackage{pgfplotstable}
\usepackage{setspace}
\usepackage{changepage}
\usepackage{caption}
\usepackage{csquotes}
\usepackage{hyperref}
\usepackage{listings}

\pgfplotsset{compat=1.18}
\hypersetup{
  colorlinks = true,
  linkcolor  = blue,
  filecolor  = magenta,      
  urlcolor   = darkgray,
  pdftitle   = semt-report-vcs-smirnov-shinakov,
}

\definecolor{codegreen}{rgb}{0,0.6,0}
\definecolor{codegray}{rgb}{0.5,0.5,0.5}
\definecolor{codepurple}{rgb}{0.58,0,0.82}
\definecolor{backcolour}{rgb}{0.99,0.99,0.99}

\lstdefinestyle{codestyle}{
  backgroundcolor=\color{backcolour},   
  commentstyle=\color{codegreen},
  keywordstyle=\color{magenta},
  numberstyle=\tiny\color{codegray},
  stringstyle=\color{codepurple},
  basicstyle=\ttfamily\footnotesize,
  breakatwhitespace=false,         
  breaklines=true,                 
  captionpos=b,                    
  keepspaces=true,                 
  numbers=left,                    
  numbersep=5pt,                  
  showspaces=false,                
  showstringspaces=false,
  showtabs=false,                  
  tabsize=2
}

\lstset{style=codestyle}
\begin{document}

\begin{titlepage}
  \begin{center}
    \begin{spacing}{1.4}
      \large{Университет ИТМО} \\
      \large{Факультет программной инженерии и компьютерной техники} \\
    \end{spacing}
    \vfill
    \textbf{
      \huge{Методы и средства программной инженерии.} \\
      \huge{Лабораторная работа №3.} \\
      \huge{Системы сборки} \\
    }
  \end{center}
  \vfill
  \begin{center}
    \begin{tabular}{r l}
      Смирнов Виктор Игоревич  & P32131 \\
      Шиняков Артём Дмитриевич & R32372 \\
      Вариант                  & 1009   \\
    \end{tabular}
  \end{center}
  \vfill
  \begin{center}
    \begin{large}
      2023
    \end{large}
  \end{center}
\end{titlepage}

\tableofcontents

\section{Задание}

Написать сценарий для утилиты Apache Ant, 
реализующий компиляцию, тестирование и 
упаковку в jar-архив кода проекта из лабораторной 
работы №3 по дисциплине "Веб-программирование".

Каждый этап должен быть выделен в отдельный блок сценария; 
все переменные и константы, используемые в сценарии, 
должны быть вынесены в отдельный файл параметров; 
MANIFEST.MF должен содержать информацию о версии и о 
запускаемом классе.

Cценарий должен реализовывать следующие цели (targets):
\begin{enumerate}
    \item compile -- компиляция исходных кодов проекта.
    \item build -- компиляция исходных кодов проекта и их 
          упаковка в исполняемый jar-архив. Компиляцию исходных 
          кодов реализовать посредством вызова цели compile.
    \item clean -- удаление скомпилированных классов проекта 
          и всех временных файлов (если они есть).
    \item test -- запуск junit-тестов проекта. Перед запуском 
          тестов необходимо осуществить сборку проекта 
          (цель build).
    \item scp - перемещение собранного проекта по scp на 
          выбранный сервер по завершению сборки. 
          Предварительно необходимо выполнить сборку проекта 
          (цель build)
    \item history - если проект не удаётся скомпилировать 
          (цель compile), загружается предыдущая версия из 
          репозитория git. Операция повторяется до тех пор, 
          пока проект не удастся собрать, либо не будет 
          получена самая первая ревизия из репозитория. 
          Если такая ревизия найдена, то формируется файл, 
          содержащий результат операции diff для всех файлов, 
          измёненных в ревизии, следующей непосредственно за 
          последней работающей.
\end{enumerate}

\section{build.properties}

\lstinputlisting[
    caption={build.properties}
]{../build.properties}

\section{build.xml}

\lstinputlisting[
    language=Xml,
    caption={build.xml}
]{../build.xml}

\section{UserAccountTest.java}

\lstinputlisting[
    language=Java,
    caption={UserAccountTest.java}
]{../src/test/UserAccountTest.java}

\section{Вывод}

Выполнив данную лабораторную работу мы научились 
познакомились с основными принципами работы 
систем автоматической сборки проектов на примере
Ant, а также поняли, как использовать фреймворк
JUnit для юнит-тестирования нашего программного продукта.

\end{document}
